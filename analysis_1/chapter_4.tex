\section{Integers and rationals}
\subsection{The integers}
\begin{definition}{A1.4.1.1}
    (Integers) An \emph{integer} is an expression of the form $a\longdash b$, where $a$ and $b$ are natural numbers.
    Two integers are considered to be equal, $a\longdash b=c\longdash d$, if and only if $a+d=c+b$. We let $\integers$ denote
    the set of all integers.
\end{definition}

\begin{definition}{A1.4.1.2}
    The sum of two integers, $a\longdash b + c\longdash d$, is defined by the formula
    \begin{align*}
        (a\longdash b) + (c\longdash d) := (a+c)\longdash(b+d).
    \end{align*}
    The product of two integers, $a\longdash b \times c\longdash d$, is defined by
    \begin{align*}
        (a\longdash b) \times (c\longdash d) := (ac+bd)\longdash(ad+bc).
    \end{align*}
\end{definition}

\begin{lemma}{A1.4.1.3}
    (Addition and multiplication are well-defined) \emph{Let $a, b, a', b', c, d$ be natural numbers.
    If $(a\longdash b)$ = $(a'\longdash b')$, then $(a\longdash b) + (c\longdash d) = (a'\longdash b') + (c\longdash d)$}
    and $(a\longdash b) \times (c\longdash d) = (a'\longdash b') \times (c\longdash d)$, and also
    $(c\longdash d) + (a\longdash b) = (c\longdash d) + (a'\longdash b')$ and
    $(c\longdash d) \times (a\longdash b) = (c\longdash d) \times (a'\longdash b')$.
    Thus addition and multiplication are well-defined operations (equal inputs give equal outputs).
\end{lemma}

\begin{definition}{A1.4.1.4}
    (Negation of integers) If $(a\longdash b)$ is an integer, we define the negation $-(a\longdash b)$ to be the integer
    $(b\longdash a)$.
    In particular if $n=n\longdash 0$ is a positive natural number, we can define its negation $-n=0\longdash n$.
\end{definition}

\begin{lemma}{A1.4.1.5}
    (Trichotomy of integers) \emph{Let $x$ be an integer. Then exactly one of the following statements is true: (a)
    $x$ is zero; (b) $x$ is equal to a positive natural number $n$; or (c) $x$ is the negation $-n$ of a positive
    natural number $n$.}
\end{lemma}

\begin{lemma}{A1.4.1.6}
    (Laws of algebra for integers) \emph{Let $x, y, z$ be integers. Then we have
    \begin{align*}
        x + y &= y + x,\\
        (x + y) + z &= x + (y + z),\\
        x + 0 = 0 + x &= x,\\
        x + (-x) = (-x) + x &= 0,\\
        xy &= yx,\\
        (xy)z &= x(yz),\\
        x1=1x &= x,\\
        x(y + z) = xy + xz &= xy + xz,\\
        (y - z)x &= yx + zx.
    \end{align*}
    }
\end{lemma}

\begin{proposition}{A1.4.1.8}
    (Integers have no zero divisors) \emph{Let $a$ and $b$ be integers such that $ab = 0$. Then either $a = 0$ or
    $b = 0$ (or both).}
\end{proposition}

\begin{corollary}{A1.4.1.9}
    (Cancellation law for integers) \emph{If $a, b, c$ are integers such that $ac = bc$ and $c$ is non-zero, then
    $a = b$.}
\end{corollary}

\begin{definition}{A1.4.1.10}
    (Ordering of the integers) Let $n$ and $m$ be integers. We say that $n$ is \emph{greater than or equal to} $m$,
    and write $n \geq m$ or $m \leq n$, if we have $n = m + a$ for some natural number $a$. We say that $n$ is
    \emph{strictly greater than m}, and write $n > m$ or $m < n$, if $n \geq m$ and $n \neq m$.
\end{definition}

\begin{lemma}{A1.4.1.11}
    (Properties of order) \emph{Let $a, b, c$ be integers.
    \begin{enumerate}
        \item $a > b$ if and only if $a - b$ is a positive natural number.
        \item (Addition preserves order) If $a > b$, then $a + c > b + c$.
        \item (Positive multiplication preserves order) If $a > b$ and $c > 0$, then $ac > bc$.
        \item (Negation reverses order) If $a > b$, then $-a < -b$.
        \item (Order is transitive) If $a > b$ and $b > c$, then $a > c$.
        \item (Order trichotomy) Exactly one of the statements $a > b$, $a = b$, or $a < b$ is true.
    \end{enumerate}
    }
\end{lemma}

\subsection{The rationals}
\begin{definition}{A1.4.2.1}
    (Rationals) A \emph{rational number} is an expression of the form $a // b$, where $a$ and $b$ are integers
    and $b$ is non-zero; $a // 0$ is not considered to be a rational number.
    Two rational numbers are considered to be equal, $a // b=c // d$, if and only if
    $ad=bc$. The set of all rational numbers is denoted $\mathbb{Q}$.
\end{definition}

\begin{definition}{A1.4.2.2}
    If $a // b$ and $c // d$ are rational numbers, we define the sum
    \begin{align*}
        (a // b) + (c // d) := (ad+bc) // (bd)
    \end{align*}
    and the product
    \begin{align*}
        (a // b) \times (c // d) := (ac) // (bd)
    \end{align*}
    and the negation
    \begin{align*}
        -(a // b) := (-a) // b.
    \end{align*}
\end{definition}

\begin{lemma}{A1.4.2.3}
    \emph{The sum, production, and negation operations on rational numbers are well-defined, in the sense that if one
    replaces $a // b$ with another rational number $a' // b'$ which is equal to $a // b$, then the output of the above
    operations remains unchanged, and similarly for $c // d$.}
\end{lemma}

\begin{proposition}{A1.4.2.4}
    (Laws of algebra for rationals) \emph{Let $x, y, z$ be rationals. Then the following laws of algebra hold:
    \begin{align*}
        x + y &= y + x,\\
        (x + y) + z &= x + (y + z),\\
        x + 0 = 0 + x &= x,\\
        x + (-x) = (-x) + x &= 0,\\
        xy &= yx,\\
        (xy)z &= x(yz),\\
        x1 = 1x &= x,\\
        x(y + z) = xy + xz &= xy + xz,\\
        (y + z)x &= yx + zx.
    \end{align*}
    If $x$ is non-zero, we also have
    \begin{align*}
        xx^{-1} = x^{-1}x = 1.
    \end{align*}
    }
\end{proposition}

\begin{definition}{A1.4.2.6}
    A rational number $x$ is said to be \emph{positive} iff we have $x = a/b$ for some positive integers $a, b$.
    It is said to be \emph{negative} iff we have $x = -y$ for some positive rational number $y$ (i.e., $x = (-a)/b$
    for some positive integers $a$ and $b$).
\end{definition}

\begin{lemma}{A1.4.2.7}
    (Ordering of the rationals). Let $x$ and $y$ be rational numbers. We say that $x > y$ iff $x - y$ is a positive
    rational number, and $x < y$ iff $x - y$ is a negative rational number. We write $x \geq y$ iff either $x > y$ or
    $x = y$, and similarly define $x \leq y$.
\end{lemma}

\begin{proposition}{A1.4.2.9}
    (Basic properties of order on the rationals) \emph{Let $x, y, z$ be rational numbers. Then the following properties
    hold:
    \begin{enumerate}
        \item (Order trichotomy) Exactly one of the statements $x = y$, $x < y$, or $x > y$ is true.
        \item (Order is anti-symmetric) One has $x < y$ if and only if $y > x$.
        \item (Order is transitive) If $x < y$ and $y < z$, then $x < z$.
        \item (Addition preserves order) If $x < y$, then $x + z < y + z$.
        \item (Positive multiplication preserves order) If $x < y$ and $z$ is positive, then $xz < yz$.
    \end{enumerate}
}
\end{proposition}

\subsection{Absolute value and exponentiation}
\begin{definition}{A1.4.3.1}
    (Absolute value) If $x$ is a rational number, the \emph{absolute value} $|x|$ of $x$ is defined as follows. If $x$ is
    positive, then $|x| := x$. If $x$ is negative, then $|x| := -x$. If $x$ is zero, then $|x| := 0$.
\end{definition}

\begin{definition}{A1.4.3.2}
    (Distance) Let $x$ and $y$ be rational numbers. The quantity $|x - y|$ is called the \emph{distance between $x$ and
    $y$} and is sometimes denoted $d(x, y)$, thus $d(x, y) = |x - y|$. For instance, $d(3, 5) = 2$.
\end{definition}

\begin{proposition}{A1.4.3.3}
    (Basic properties of absolute value and distance) \emph{Let $x, y, z$ be rational numbers.
    \begin{enumerate}
        \item (Non-degeneracy of absolute value) We have $|x| \geq 0$. Also, $|x| = 0$ if and only if $x = 0$.
        \item (Triangle inequality for absolute value) We have $|x + y| \leq |x| + |y|$.
        \item We have inequalities $-y \leq x \leq y$ if and only if $y \geq |x|$. In particular, we have
        $-|x| \leq x \leq |x|$.
        \item (Multiplicativity of absolute value) We have $|xy| = |x||y|$. In particular, $|-x| = |x|$.
        \item (Non-degeneracy of distance) We have $d(x, y) \geq 0$. Also, $d(x, y) = 0$ if and only if $x = y$.
        \item (Symmetry of distance) We have $d(x, y) = d(y, x)$.
        \item (Triangle inequality for distance) We have $d(x, z) \leq d(x, y) + d(y, z)$.
    \end{enumerate}
    }
\end{proposition}

\begin{definition}{A1.4.3.4}
    ($\epsilon$ -closeness). Let $\epsilon > 0$ be a rational number, and let $x, y$ be rational numbers. We say that $y$
    is \emph{$\epsilon$ -close} to $x$ iff we have $d(x, y) < \epsilon$.
\end{definition}

\begin{proposition}{A1.4.3.7}
    \emph{Let $x, y, z$ be rational numbers.
    \begin{enumerate}
        \item If $x = y$, then $x$ is $\epsilon$-close to $y$ for every $\epsilon > 0$. Conversely, if $x$ is $\epsilon$-close
        to $y$ for every $\epsilon > 0$, then $x = y$.
        \item Let $epsilon > 0$. If $x$ is $\epsilon$-close to $y$, then $y$ is $\epsilon$-close to $x$.
        \item Let $\epsilon, \delta > 0$. If $x$ is $\epsilon$-close to $y$ and $y$ is $\delta$-close to $z$, then $x$
        and $z$ are $(\epsilon + \delta)$-close.
        \item Let $\epsilon, \delta > 0$. If $x$ and $y$ are $\epsilon$-close, and $z$ and $w$ are $\delta$-close, then
        $x + z$ and $y + w$ are $(\epsilon + \delta)$-close, and $x - z$ and $y - w$ are also $(\epsilon + \delta)$-close.
        \item Let $\epsilon > 0$. If $x$ and $y$ are $\epsilon$-close, they are also $\epsilon ' $-close for every
        $\epsilon ' > \epsilon$.
        \item Let $\epsilon > 0$. If $y$ and $z$ are both $\epsilon$-close to $x$, and $w$ is between $y$ and $z$
        (i.e., $y \leq w \leq z$ or $z \leq w \leq y$), then $w$ is also $\epsilon$-close to $x$.
        \item Let $\epsilon > 0$. If $x$ and $y$ are $\epsilon$-close, and $z$ is non-zero, then $xz$ and $yz$ are
        $\epsilon z$-close.
        \item Let $\epsilon , \delta > 0$. If $x$ and $y$ are $\epsilon$-close, and $z$ and $w$ are $\delta$-close,
        then $xz$ and $yw$ are $(\epsilon |z| + \delta |w| + \epsilon \delta)$-close.
    \end{enumerate}
    }
\end{proposition}

\begin{definition}{A1.4.3.9}
    (Exponentiation to a natural number) Let $x$ be a rational number. To raise $x$ to the power 0, we define $x^0 := 1$;
    in particular we define $0^0 := 1$. Now suppose inductively that we have defined how to raise $x$ to the power $n$.
    Then we can raise $x$ to the power $n++$ by defining $x^{n++} := x^n \times x$.
\end{definition}

\begin{proposition}{A1.4.3.10}
    (Properties of exponentiation, I). Let $x, y$ be rational numbers, and let $n, m$ be natural numbers.
    \begin{enumerate}
        \item We have $x^n x^m = x^{n + m}, (x^n)^m = x^{nm}, $ and $(xy)^n = x^n y^n$.
        \item Suppose $n > 0$. Then we have $x^n = 0$ if and only if $x = 0$.
        \item If $x \geq y \geq 0$, then $x^n \geq y^n \geq 0$. If $x > y \geq 0$ and $n > 0$, then $x^n > y^n \geq 0$.
        \item We have $|x^n| = |x|^n$.
    \end{enumerate}
\end{proposition}

\begin{definition}{A1.4.3.11}
    (Exponentiation to a negative number) Let $x$ be a non-zero rational number. Then for any negative integer $-n$, we
    define $x^{-n} := 1/x^n$.
\end{definition}

\begin{proposition}{A1.4.3.12}
    (Properties of exponentiation, II) \emph{Let $x, y$ be rational numbers, and let $n, m$ be integers.
    \begin{enumerate}
        \item We have $x^n x^m = x^{n+m}, (x^n)^m = x^{nm},$ and $(xy)^n = x^n y^n$.
        \item If $x \geq y > 0$, then $x^n \geq y^n > 0$ if $n$ is positive, and $0 < x^n \leq y^n$ if $n$ is negative.
        \item If $x, y > 0, n \neq 0$, and $x^n = y^n$, then $x = y$.
        \item We have $|x^n| = |x|^n$
    \end{enumerate}
    }
\end{proposition}

\subsection{Gaps in the rational numbers}
\begin{proposition}{A1.4.4.1}
    (Interspersing of integers by rationals) \emph{Let $x$ be a rational number. Then there exists an integer $n$ such
    that $n \leq x < n + 1$. In fact, this integer is unique (i.e., for each $x$ there is only one $n$ for which
    $n \leq x < n + 1$). In particular, there exists a natural number $N$ such that $N > x$ (i.e., there is no such
    thing as a rational number which is larger than all the natural numbers).}
\end{proposition}

\begin{proposition}{A1.4.4.3}
    (Interspersing of rationals by rations) \emph{If $x$ and $y$ are two rationals
    such that $x < y$, then there exists a third rational number $z$ such that $x < z < y$.}
\end{proposition}

\begin{proposition}{A1.4.4.4}
    \emph{There does not exist a rational number $x$ such that $x^2 = 2$.}
\end{proposition}

\begin{proposition}{A1.4.4.5}
    \emph{For every rational number $\epsilon > 0$, there exists a non-negative
    rational number $x$ such that $x^2 < 2 < (x + \epsilon)^2$.}
\end{proposition}

