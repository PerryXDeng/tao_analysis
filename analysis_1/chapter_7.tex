\section{Series}
\subsection{Finite series}
\begin{definition}{A1.7.1.1}
    (Finite series) Let $m, n$ be integers, and let $(a_i)_{i=m}^{\infty}$ be a
    finite sequence of real numbers, assigning a real number $a_i$ to each
    integer $i$ between $m$ and $n$ inclusive (i.e. $m \leq i \leq n$). Then
    we define the finite sum (or finite series) $\sum_{i=m}^{n} a_i$ by the
    recursive formula
    \begin{align*}
        \sum_{i=m}^{m} a_i &:= 0 \text{whenever $n < m$;} \\
        \sum_{i=m}^{n+1} a_i &:= \sum_{i=m}^{n} a_i + a_{n+1} \text{whenever $n > m - 1$.}
    \end{align*}
\end{definition}

\begin{lemma}{A1.7.1.4}
    \emph{
    \begin{enumerate}
        \item Let $m \leq n < p$ be integers, and let $a_i$ be a real number
        assigned to each integer $m \leq i \leq p$. Then we have
        \begin{equation*}
            \sum_{i=m}^{n} a_i + \sum_{i=n+1}^{p} a_i = \sum_{i=m}^{p} a_i.
        \end{equation*}
        \item Let $m \leq n$ be integers, $k$ be another integer, and let $a_i$
        be a real number assigned to each integer $m \leq i \leq n$. Then we have
        \begin{equation*}
            \sum_{i=m}^{n} a_i = \sum_{j=m+k}^{n+k} a_{j-k}.
        \end{equation*}
        \item Let $m \leq n$ be integers, and let $a_i, b_i$ be real numbers assigned
        to each integer $m \leq i \leq n$. Then we have
        \begin{equation*}
            \sum_{i=m}^{n} (a_i + b_i) = (\sum_{i=m}^{n} a_i) + (\sum_{i=m}^{n} b_i).
        \end{equation*}
        \item Let $m \leq n$ be integers, and let $a_i$ be a real number assigned
        to each integer $m \leq i \leq n$, and let $c$ be another real number. Then
        have
        \begin{equation*}
            \sum_{i=m}^{n} (ca_i) = c(\sum_{i=m}^{n} a_i).
        \end{equation*}
        \item (Triangle inequality for finite series) Let $m \leq n$ be integers, and
        let $a_i$ be a real number assigned to each integer $m \leq i \leq n$. Then
        we have
        \begin{equation*}
            |\sum_{i=m}^{n} a_i| \leq \sum_{i=m}^{n} |a_i|.
        \end{equation*}
        \item (Comparison test for finite series) Let $m \leq n$ be integers, and
        let $a_i, b_i$ be real numbers assigned to each integer $m \leq i \leq n$.
        Suppose that $a_i \leq b_i$ for all $m \leq i \leq n$. Then we have
        \begin{equation*}
            \sum_{i=m}^{n} a_i \leq \sum_{i=m}^{n} b_i.
        \end{equation*}
    \end{enumerate}
    }
\end{lemma}

\begin{definition}{A1.7.1.6}
    (Summation over finite sets) Let $X$ be a finite set with $n$ elements (where
    $n \in \naturals$), and let $f: X \rightarrow \reals$ be a function from $X$
    to the real numbers (i.e., $f$ assigns a real number $f(x)$ to each element
    $x$ of $X$). Then we can define the finite sum $\sum_{x \in X} f(x)$ as follows.
    We first select any bijection $g$ from $\{i \in \naturals : 1 \leq i \leq n\}$
    to $X$; such a bijection exists since $X$ is assumed to have n elements. We
    then define
    \begin{equation*}
        \sum_{x \in X} f(x) := \sum_{i=1}^{n} f(g(i)).
    \end{equation*}
\end{definition}

\begin{proposition}{A1.7.1.7}
    (Summation over finite sets is well-defined) \emph{Let $X$ be a finite set
    with $n$ elements, let $f: X \rightarrow \reals$ be a function, and let $g: \{i \in \naturals : 1 \leq i \leq n\} \rightarrow X$
    and $h: \{i \in \naturals : 1 \leq i \leq n\} \rightarrow X$ be bijections.
    Then we have
    \begin{equation*}
        \sum_{x \in X} f(x) = \sum_{i=1}^{n} f(g(i)) = \sum_{i=1}^{n} f(h(i)).
    \end{equation*}}
\end{proposition}

\begin{proposition}{A1.7.1.11}
    (Basic properties of summation over finite sets) \emph{
    \begin{enumerate}
        \item If $X$ is empty, and $f: X \rightarrow \reals$ is a function (i.e.,
        $f$ is the empty function), we have
        \begin{equation*}
            \sum_{x \in X} f(x) = 0.
        \end{equation*}
        \item If $X$ consists of a single element, $X = \{x_0\}$, and $f: X \rightarrow \reals$
        is a function, we have
        \begin{equation*}
            \sum_{x \in X} f(x) = f(x_0).
        \end{equation*}
        \item (Substitution, part I) If $X$ is a finite set, $f: X \rightarrow \reals$
        is a function, and $g: Y \rightarrow X$ is a bijection, then
        \begin{equation*}
            \sum_{x \in X} f(x) = \sum_{y \in Y} f(g(y)).
        \end{equation*}
        \item (Substitution, part II) Let $n \leq m$ be integers, and let $X$ be
        the set $X := {i \in \naturals : n \leq i \leq m}$. If $a_i$ is a real number
        assigned to each integer $i \in X$, then we have
        \begin{equation*}
            \sum_{i=n}^{m} a_i = \sum_{i=1}^{m-n+1} a_{n+i-1}.
        \end{equation*}
        \item Let $X, Y$ be disjoint finite sets (so $X \cap Y = \emptyset$), and
        $f: X \cup Y \rightarrow \reals$ is a function. Then we have
        \begin{equation*}
            \sum_{z \in X \cup Y} f(z) = \sum_{x \in X} f(x) + \sum_{y \in Y} f(y).
        \end{equation*}
        \item (Linearity, part I) Let $X$ be a finite set, and let $f: X \rightarrow \reals$
        and $g: X \rightarrow \reals$ be functions. Then
        \begin{equation*}
            \sum_{x \in X} (f(x) + g(x)) = \sum_{x \in X} f(x) + \sum_{x \in X} g(x).
        \end{equation*}
        \item (Linearity, part II) Let $X$ be a finite set, let $f: X \rightarrow \reals$
        be a function, and let $c$ be a real number. Then
        \begin{equation*}
            \sum_{x \in X} cf(x) = c \sum_{x \in X} f(x).
        \end{equation*}
        \item (Monothonicity) Let $X$ be a finite set, and let $f: X \rightarrow \reals$
        and $g: X \rightarrow \reals$ be functions such that $f(x) \leq g(x)$ for all
        $x \in X$. Then we have
        \begin{equation*}
            \sum_{x \in X} f(x) \leq \sum_{x \in X} g(x).
        \end{equation*}
        \item (Triangle inequality) Let $X$ be a finite set, and let $f: X \rightarrow \reals$
        be a function, then
        \begin{equation*}
            |\sum_{x \in X} f(x)| \leq \sum_{x \in X} |f(x)|.
        \end{equation*}
    \end{enumerate}}
\end{proposition}

\begin{lemma}{A1.7.1.13}
    \emph{Let $X, Y$ be finite sets, and let $f: X \times Y \rightarrow \reals$
    be a function. Then
    \begin{equation*}
        \sum_{x \in X} \sum_{y \in Y} f(x, y) = \sum_{y \in Y} \sum_{x \in X} f(x, y).
    \end{equation*}}
\end{lemma}

\begin{corollary}{A1.7.1.14}
    (Fubini's theorem for finite series) \emph{Let $X, Y$ be finite sets, and let
    $f: X \times Y \rightarrow \reals$ be a function. Then
    \begin{align*}
        \sum_{x \in X} (\sum_{y \in Y} f(x, y)) &= \sum_{(x, y) \in X \times Y} f(x, y) \\
        &= \sum_{(y, x) \in Y \times X} f(x, y) \\
        &= \sum_{y \in Y} (\sum_{x \in X} f(x, y)).
    \end{align*}}
\end{corollary}

\subsection{Infinite series}
\begin{definition}{A1.7.2.1}
    (Formal infinite series) A (formal) infinite series is any expression of the
    form
    \begin{equation*}
        \sum_{n=m}^{\infty} a_n
    \end{equation*}
    where $m$ is an integer, and $a_n$ is a real number for any integer $n \geq m$.

    We sometimes write this series as
    \begin{equation*}
        a_m + a_{m+1} + a_{m+2} + \ldots
    \end{equation*}
\end{definition}

\begin{definition}{A1.7.2.2}
    (Convergence of series) Let $\sum_{n=m}^{\infty} a_n$ be a formal infinite series.
    For any integer $N \geq m$, we define the \emph{$N^{th}$ partial sum} $S_N$ of
    this series to be $S_N := \sum_{n=m}^{N} a_n$.; of course, $S_N$ is a real number.
    If the sequence $(S_N)_{N=m}^{\infty}$ converges to some limit $L$ as $N \rightarrow \infty$,
    then we say that the infinite series $\sum_{n=m}^{\infty} a_n$ is \emph{convergent},
    and \emph{converges to $L$}; we also write $L = \sum_{n=m}^{\infty} a_n$, and
    say that $L$ is the \emph{sum} of  the infinite series $\sum_{n=m}^{\infty} a_n$.
    If the partial sums $(S_N)_{N=m}^{\infty}$ diverge, then we say that the infinite
    series $\sum_{n=m}^{\infty} a_n$ is \emph{divergent}, and we do not assign any
    real number value to that series.
\end{definition}

\begin{proposition}{A1.7.2.5}
    \emph{Let $\sum_{n=m}^{\infty} a_n$ be a formal series of real numbers. Then
    $\sum_{n=m}^{\infty} a_n$ converges if and only if, for every real number
    $\epsilon > 0$, there exists an integer $N \geq m$ such that
    \begin{equation*}
        |\sum_{n=p}^{q} a_n| < \epsilon
    \end{equation*}
    }
\end{proposition}
