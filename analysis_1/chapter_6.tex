\section{Limits of sequences}
\subsection{Convergence and limit laws}
\begin{definition}{A6.1.1}
    (Distance between two real numbers) Given two real numbers $x$ and $y$, we
    define their distance $d(x, y)$ to be $d(x, y) := |x - y|$.
\end{definition}

\begin{definition}{A6.1.2}
    ($\epsilon$-close real numbers) Let $\epsilon > 0$ be a real number. We say that
    two real numbers $x, y$ are \emph{$\epsilon$-close} iff we have $d(x, y) \leq \epsilon$.
\end{definition}

\begin{definition}{A1.6.1.3}
    (Cauchy sequences of reals) Let $\epsilon > 0$ be a real number. A sequence
    $(a_n)_{n=N}^{\infty}$ starting at some integer $N$ is said to be \emph{$\epsilon$-steady}
    iff $a_j$ and $a_k$ are $epsilon$-close for every $j, k \geq N$. A sequence
    $(a_n)_{n=m}^{\infty}$ is said to be \emph{eventually $\epsilon$-steady} iff
    there exists an $N \geq m$ such that $(a_n)_{n=N}^{\infty}$ is $\epsilon$-steady.
    We say that $(a_n)_{n=m}^{\infty}$ is a \emph{Cauchy sequence} iff it is
    eventually $\epsilon$-steady for every $\epsilon > 0$.
\end{definition}

\begin{proposition}{A1.6.1.4}
    \emph{Let $(a_n)_{n=m}^{\infty}$ be a sequence of rational numbers starting
    at some integer $m$. Then $(a_n)_{n=m}^{\infty}$ is a Cauchy sequence in the
    sense of Definition 5.1.8 if and only if it is a Cauchy sequence in the sense
    of Defitnition 6.1.3.}
\end{proposition}

\begin{definition}{A1.6.1.5}
    (Convergence of sequences) Let $\epsilon > 0$ be a real number, and let $L$
    be a real number. A sequence $(a_n)_{n=N}^{\infty}$ of real numbers is said
    to be $\epsilon$-close to $L$ iff $a_n$ is $epsilon$-close to $L$ for every
    $n \geq N$, i.e., we have $|a_n - L| \leq \epsilon$ for every $n \geq N$.
    We say that a sequence $(a_n)_{n=m}^{\infty}$ is \emph{eventually $\epsilon$-close}
    to $L$ iff there exists an $N \geq m$ such that $(a_n)_{n=N}^{\infty}$
    is $epsilon$-close to $L$. We say that a sequence $(a_n)_{n=m}^{\infty}$
    \emph{converges} to $L$ iff it is eventually $epsilon$-close to $L$ for every
    $\epsilon > 0$.
\end{definition}

\begin{proposition}{A1.6.1.7}
    (Uniqueness of limits) \emph{Let $(a_n)_{n=m}^{\infty}$ be a real sequence
    starting at some integer index $m$, and let $L \neq L'$ be two distinct real
    numbers. THen it is not possible for $(a_n)_{n=m}^{\infty}$ to converge to $L$
    while also converging to $L'$.}
\end{proposition}

\begin{definition}{A1.6.1.8}
    (Limits of sequences) If a sequence $(a_n)_{n=m}^{\infty}$ converges to some
    real number $L$, we say that $(a_n)_{n=m}^{\infty}$ is \emph{convergent} and
    that its \emph{limit} is $L$; we write
    \begin{equation*}
        L = \limit{a_n}
    \end{equation*}
    to denote this fact. If a sequence $(a_n)_{n=m}^{\infty}$ is not converging
    to any real number $L$, we say that the sequence $(a_n)_{n=m}^{\infty}$ is
    \emph{divergent} and we leave $\limit{a_n}$ undefined.
\end{definition}

\begin{proposition}{A1.6.1.10}
    \emph{We have $\limit(1/n) = 0$}.
\end{proposition}

\begin{proposition}{A1.6.1.12}
    (Convergent sequences are Cauchy) \emph{Suppose that $(a_n)_{n=m}^{\infty}$
    is a convergent sequence of real numbers. Then $(a_n)_{n=m}^{\infty}$ is also
    a Cauchy sequence.}
\end{proposition}

\begin{proposition}{A1.6.1.15}
    (Formal limits are genuine limits) \emph{Suppose that $(a_n)_{n=1}^{\infty}$
    is a Cauchy sequence
    of rational numbers. Then $(a_n)_{n=1}^{\infty}$ converges
    to $\formallimit(a_n)$, i.e.
    \begin{equation*}
        \formallimit{a_n} = \limit{a_n}.
    \end{equation*}
    }
\end{proposition}

\begin{definition}{A1.6.1.16}
    (Bounded sequences) A sequence $(a_n)_{n=m}^{\infty}$ of real numbers is
    \emph{bounded} by a real number $M$ iff we have $|a_n| \leq M$
    for all $n \geq m$. We say that $(a_n)_{n=m}^{\infty}$ is \emph{bounded} iff
    it is bounded for some real number $M > 0$.
\end{definition}

\begin{corollary}{A1.6.1.17}
    \emph{Every convergent sequence of real numbers is bounded}.
\end{corollary}

\begin{theorem}{A1.6.1.19}
    (Limit Laws) Let $(a_n)_{n=m}^{\infty}$ and $(b_n)_{n=m}^{\infty}$ be convergent
    sequences of real numbers, and let $x, y$ be the real numbers $x := \limit{a_n}$
    and $y:= \limit{b_n}$.
    \begin{enumerate}
        \item The sequences $(a_n + b_n)_{n=m}^{\infty}$ converges to $x + y$; in other
        words,
        \begin{equation*}
            \limit{a_n + b_n} = \limit{a_n} + \limit{b_n}
        \end{equation*}
        \item The sequence $(a_n b_n)_{n=m}^{\infty}$ converges to $xy$; in other
        words,
        \begin{equation*}
            \limit{a_n b_n} = (\limit{a_n})(\limit{b_n})
        \end{equation*}
        \item For any real number $c$, the sequence $(ca_n)_{n=m}^{\infty}$ converges
        to $cx$; in other words,
        \begin{equation*}
            \limit{ca_n} = c\limit{a_n}
        \end{equation*}
        \item The sequence $(a_n - b_n)_{n=m}^{\infty}$ converges to $x - y$;
        in other words,
        \begin{equation*}
            \limit{a_n - b_n} = \limit{a_n} - \limit{b_n}
        \end{equation*}
        \item Suppose that $y \neq 0$, and that $b_n \neq 0$ for all $n \geq m$.
        Then the sequence $(b^{-1}_n)_{n=m}^{\infty}$ converges to $y^{-1}$; in
        other words,
        \begin{equation*}
            \limit{b^{-1}_n} = (\limit{b_n})^{-1}
        \end{equation*}
        \item Suppose that $y \neq 0$, and that $b_n \neq 0$ for all $n \geq m$.
        Then the sequence $(a_n / b_n)_{n=m}^{\infty}$ converges to $x/y$; in other
        words,
        \begin{equation*}
            \limit{a_n / b_n} = \frac{\limit{a_n}}{\limit{b_n}}
        \end{equation*}
        \item The sequence $(\max(a_n, b_n))_{n=m}^{\infty}$ converges to $\max(x,y)$;
        in other words,
        \begin{equation*}
            \limit{\max(a_n / b_n)} = \max(\limit{a_n}, \limit{b_n})
        \end{equation*}
        \item The sequence $(\min(a_n, b_n))_{n=m}^{\infty}$ converges to $\min(x,y)$;
        in other words,
        \begin{equation*}
            \limit{\min(a_n / b_n)} = \min(\limit{a_n}, \limit{b_n})
        \end{equation*}
    \end{enumerate}
\end{theorem}

\subsection{The extended real number system}
\begin{definition}{A1.6.2.1}
    (Extended real number system) The \emph{extended real number system} $\extendedreals$
    is the real line $\reals$ with two additional elements attached, called
    $+\infty$ and $-\infty$. These elements are distinct from each other and also
    distinct from every real number. An extended real number $x$ is called \emph{finite}
    if it is a real number, and \emph{infinite} iff it is equal to $+\infty$ or
    $-\infty$. (This definition is not directly related to the notion of finite
    and infinite sets in Section 3.6, though it is of course similar in spirit.)
\end{definition}

\begin{definition}{A1.6.2.2}
    (Negation of extended reals) The operation of negation $x \mapsto -x$ on $\reals$,
    we now extend to $\extendedreals$ by defining $-(+\infty) := -\infty$ and
    $-(-\infty) := +\infty$.

    Thus every extended real number $x$ has a negation, and $-(-x)$ is always equal
    to $x$.
\end{definition}

\begin{definition}{A1.6.2.3}
    (Ordering of extended reals) Let $x$ and $y$ be extended real numbers. We say
    that $x \leq y$, i.e., $x$ is less than or equal to $y$, iff one of the following
    three statements is true:
    \begin{enumerate}
        \item $x$ and $y$ are real numbers, and $x \leq y$ as real numbers
        \item $y = +\infty$
        \item $x = -\infty$
    \end{enumerate}
    We say that $x < y$ if we have $x \leq y$ and $x \neq y$. We sometimes write
    $x < y$ as $y > x$, and $x \leq y$ as $y \geq x$.
\end{definition}

\begin{proposition}{A1.6.2.5}
    \emph{Let $x, y, z$ be extended real numbers. Then the following statements
    are true:
    \begin{enumerate}
        \item (Reflexivity) We have $x \leq x$
        \item (Trichotomy) Exactly one of the statements $x < y$, $x = y$, or $x > y$
        is true
        \item (Transitivity) If $x \leq y$ and $y \leq z$, then $x \leq z$
        \item (Negation reverses order) If $x \leq y$, then $-y \leq -x$
    \end{enumerate}
    }
\end{proposition}

\begin{definition}{A1.6.2.6}
    (Supremum of sets of extended reals) Let $E$ be a subset of $\extendedreals$.
    Then we define the \emph{supremum} $\supremum{E}$ or \emph{least upper bound}
    of $E$ by the following rule.
    \begin{enumerate}
        \item If $E$ is contained in $\reals$ (i.e., $+\infty$ and $-\infty$ are not
        elements of $E$), then we let $\supremum{E}$ be as defined in Definition 5.5.10
        \item If $E$ contains $+\infty$, then we set $\supremum{E} := +\infty$
        \item If $E$ does not contain $+\infty$, but does contain $-\infty$, then we
        set $\supremum{E} := \supremum{E \setminus \{-\infty\}}$ (which is a subset of $\reals$
        and thus falls under case (a))
    \end{enumerate}
    We also define the \emph{infimum} $\infimum{E}$ of $E$ (also known as the \emph{greatest
    lower bound} of $E$) by the formula
    \begin{equation*}
        \infimum{E} := -\supremum{-E}
    \end{equation*}
    where $-E$ is the set $\{-x: x \in E\}$.
\end{definition}

\begin{theorem}{A1.6.2.11}
    \emph{Let $E$ be a subset of $\extendedreals$. Then the following statements
    are true.
    \begin{enumerate}
        \item For every $x \in E$, we have $x \leq \supremum{E}$ and $x \geq \infimum{E}$
        \item Suppose that $M \in \extendedreals$ is an upper bound for $E$, i.e.,
        $x \leq M$ for all $x \in E$. Then we have $\supremum{E} \leq M$.
        \item Suppose that $m \in \extendedreals$ is a lower bound for $E$, i.e.,
        $x \geq m$ for all $x \in E$. Then we have $\infimum{E} \geq m$.
    \end{enumerate}
    }
\end{theorem}
        
\subsection{Suprema and Infima of sequences}
\begin{definition}{A1.6.3.1}
    (Sup and inf of sequences) Let $(a_n)_{n=m}^{\infty}$ be a sequence of real
    numbers. Then we define $\supremum{a_n}_{n=m}^{\infty}$ to be the supremum of the set
    $\{a_n: n \geq m\}$, and $\infimum{a_n}_{n=m}^{\infty}$ to be the infimum of the same set
    $\{a_n: n \geq m\}$.
\end{definition}

\begin{proposition}{A1.6.3.6}
    (Least upper bound property) \emph{Let $(a_n)_{n=m}^{\infty}$ be a sequence
    of real numbers, and let $x$ be the extended real number $x := \supremum{a_n}_{n=m}^{\infty}$.
    Then we have $a_n \leq x$ for all $n \geq m$. Also, whenever $M \in \extendedreals$
    is an upper bound for $a_n$ (i.e., $a_n \leq M$ for all $n \geq m$), we have
    $x \leq M$. Finally, for every extended real number $y$ for which $y < x$,
    there exists at least one $n \geq m$ for which $y < a_n \leq x$.}
\end{proposition}

\begin{proposition}{A1.6.3.8}
    (Monotone bounded sequences converge) \emph{Let $(a_n)_{n=m}^{\infty}$ be a
    sequence of real numbers which has some finite upper bound $M \in \reals$,
    and which is also increasing (i.e., $a_{n+1} \geq a_n$ for all $n \geq m$).
    Then the sequence $(a_n)_{n=m}^{\infty}$ converges to some real number $L$.
    Then $(a_n)_{n=m}^{\infty}$ is convergent, and in fact
    \begin{equation*}
        \limit{a_n} = \supremum{a_n}_{n=m}^{\infty} \leq M
    \end{equation*}
    }
\end{proposition}

\begin{proposition}{A1.6.3.10}
    \emph{Let $0 < x < 1$. Then we have $\limit{x^n = 0}$.}
\end{proposition}

\subsection{Limsup, Liminf, and limit points}
\begin{definition}{A1.6.4.1}
    (Limit points) Let $(a_n)_{n=m}^{\infty}$ be a sequence of real numbers, let
    $x$ be a real number, and let $\epsilon > 0$ be a real number. We say that $x$
    is \emph{$\epsilon$-adherent} to $(a_n)_{n=m}^{\infty}$ iff there exists an
    $n \geq m$ such that $a_n$ is $\epsilon$-close to $x$. We say that $x$ is a
    \emph{limit point} or \emph{adherent point} of $(a_n)_{n=m}^{\infty}$ iff it
    is continually $\epsilon$-adherent to $(a_n)_{n=m}^{\infty}$ for every $\epsilon > 0$.
\end{definition}

\begin{proposition}{A1.6.4.5}
    (Limits are limit points) \emph{Let $(a_n)_{n=m}^{\infty}$ be a sequence which
    converges to a real number $c$. Then $c$ is a limit point of $(a_n)_{n=m}^{\infty}$,
    and in fact, $c$ is the only limit point of $(a_n)_{n=m}^{\infty}$.}
\end{proposition}

\begin{definition}{A1.6.4.6}
    (Limit superior and limit inferior) Suppose that $(a_n)_{n=m}^{\infty}$ is a
    sequence. We define a new sequence $(a_N^+)_{N=m}^{\infty}$ by the formula
    \begin{equation*}
        a_N^+ := \supremum{a_n}_{n=N}^{\infty}
    \end{equation*}
    More informally, $a_N^+$ is the supremum of all the elements of the sequence
    from $a_N$ onwards. We then define the \emph{limit superior} of the sequence
    $(a_n)_{n=m}^{\infty}$, denoted $\limitsuperior{a_n}$, by the formula
    \begin{equation*}
        \limitsuperior{a_n} := \infimum{a_N^+}_{N=m}^{\infty}
    \end{equation*}
    Similarly, we can define
    \begin{equation*}
        a_N^- := \infimum{a_n}_{n=N}^{\infty}
    \end{equation*}
    and define the \emph{limit inferior} of the sequence $(a_n)_{n=m}^{\infty}$,
    denoted $\limitinferior{a_n}$, by the formula
    \begin{equation*}
        \limitinferior{a_n} := \supremum{a_N^-}_{N=m}^{\infty}
    \end{equation*}
\end{definition}

\begin{proposition}{A1.6.4.12}
    \emph{Let $(a_n)_{n=m}^{\infty}$ be a sequence of real numbers, let $L^+$ be
    the limit superior of this sequence, and let $L^-$ be the limit inferior of
    this sequence (thus $L^+$ and $L^-$ are extended real numbers).
    \begin{enumerate}
        \item For every $x > L+$, there exists an $N \geq m$ such that $a_n < x$
        for all $n \geq N$. (In other words, for every $x > L^+$, the elements of
        the sequence $(a_n)_{n=m}^{\infty}$ are eventually less than $x$.) Similarly,
        for every $y < L^-$ there exists an $N \geq m$ such that $a_n > y$ for all
        $n \geq N$.
        \item For every $x < L^+$, and every $N \geq m$, there exists an $n \geq N$
        such that $a_n > x$. (In other words, for every $x < L^+$, the elements of
        the sequence $(a_n)_{n=m}^{\infty}$ exceed $x$ infinitely often.) Similarly,
        for every $y > L^-$, and every $N \geq m$, there exists an $n \geq N$ such
        that $a_n < y$.
        \item We have $\infimum{a_n}_{n=m}^{\infty} \leq L^- \leq L^+ \leq \supremum{a_n}_{n=m}^{\infty}$.
        \item If $c$ is any limit point of $(a_n)_{n=m}^{\infty}$, then we have
        $L^- \leq c \leq L^+$.
        \item If $L^+$ is finite, then it is a limit point of $(a_n)_{n=m}^{\infty}$.
        Similarly, if $L^-$ is finite, then it is a limit point of $(a_n)_{n=m}^{\infty}$.
        \item Let $c$ be a real number. If $(a_n)_{n=m}^{\infty}$ converges to $c$
        , then we must have $L^+ = L^- = c$. Conversely, if $L^+ = L^- = c$, then
        $(a_n)_{n=m}^{\infty}$ converges to $c$.
    \end{enumerate}
    }
\end{proposition}

\begin{lemma}{A1.6.4.13}
    (Comparison principle) \emph{Suppose that $(a_n)_{n=m}^{\infty}$ and $(b_n)_{n=m}^{\infty}$
    are two sequences of real numbers such that $a_n \leq b_n$ for all $n \geq m$.
    Then we have the inequalities
    \begin{align*}
        \supremum{a_n}_{n=m}^{\infty} &\leq \supremum{b_n}_{n=m}^{\infty} \\
        \infimum{a_n}_{n=m}^{\infty} &\leq \infimum{b_n}_{n=m}^{\infty} \\
        \limitsuperior{a_n} &\leq \limitsuperior{b_n} \\
        \limitinferior{a_n} &\leq \limitinferior{b_n}
    \end{align*}
    }
\end{lemma}

\begin{corollary}{A1.6.4.14}
    (Squeeze test) \emph{Let $(a_n)_{n=m}^{\infty}$, $(b_n)_{n=m}^{\infty}$, and
    $(c_n)_{n=m}^{\infty}$ be sequences of real numbers such that
    \begin{equation*}
        a_n \leq b_n \leq c_n
    \end{equation*}
    for all $n \geq m$. Suppose also that $(a_n)_{n=m}^{\infty}$ and $(c_n)_{n=m}^{\infty}$
    both converge to the same limit $L$. Then $(b_n)_{n=m}^{\infty}$ is also convergent
    to $L$.}
\end{corollary}

\begin{corollary}{A1.6.4.17}
    (Zero test for sequences) \emph{Let $(a_n)_{n=m}^{\infty}$ be a sequence of real
    numbers. Then the limit $\limit{a_n}$ exists and is equal to 0 if and only if
    the limit $\limit{|a_n|}$ exists and is equal to 0.}
\end{corollary}

\begin{theorem}{A1.6.4.18}
    (Completeness of the reals) \emph{A sequence $(a_n)_{n=m}^{\infty}$ of real numbers
    is a Cauchy sequence if and only if it is convergent.}
\end{theorem}

\subsection{Some standard limits}
\begin{corollary}{A1.6.5.1}
    \emph{We have $\limit{1/n^{1/k}} = 0$ for every integer $k \geq 1$.}
\end{corollary}

\begin{lemma}{A1.6.5.2}
    \emph{Let $x$ be a real number. Then we have $\limit{x^n}$ exists and is equal
    to zero when $|x| < 1$, exists and is equal to $1$ when $x = 1$, and diverges
    when $x < -1$ or when $|x| > 1$.}
\end{lemma}

\begin{lemma}{A1.6.5.3}
    \emph{For any $x > 0$, we have $\limit{x^{1/n}} = 1$}
\end{lemma}

\subsection{Subsequences}
\begin{definition}{A1.6.6.1}
    (Subsequences) Let $(a_n)_{n=m}^{\infty}$ and $(b_n)_{n=N}^{\infty}$ be sequences
    of real numbers. We say that $(b_n)_{n=N}^{\infty}$ is a \emph{subsequence} of
    $(a_n)_{n=m}^{\infty}$ iff there exists a function $f: \naturals \to \naturals$
    which is strictly increasing (i.e., $f(n + 1) > f(n)$ for all $n \in \naturals$)
    such that
    \begin{equation*}
        b_n = a_{f(n)} \text{for all} n \in \naturals.
    \end{equation*}
\end{definition}

\begin{lemma}{A1.6.6.4}
    \emph{Let $(a_n)_{n=m}^{\infty}$, $(b_n)_{n=N}^{\infty}$, and $(c_n)_{n=P}^{\infty}$
    be sequences of real numbers. Then $(a_n)_{n=m}^{\infty}$ is the subsequence
    of $(a_n)_{n=m}^{\infty}$. Furthermore, if $(b_n)_{n=N}^{\infty}$ is a subsequence
    of $(a_n)_{n=m}^{\infty}$, and $(c_n)_{n=P}^{\infty}$ is a subsequence of
    $(b_n)_{n=N}^{\infty}$, then $(c_n)_{n=P}^{\infty}$ is a subsequence of
    $(a_n)_{n=m}^{\infty}$.}
\end{lemma}

\begin{proposition}{A1.6.6.5}
    (Subsequences related to limits) \emph{Let $(a_n)_{n=m}^{\infty}$ be a sequence
    of real numbers, and let $L$ be a real number. Then the following two statements
    are logically equivalent (each one implies the other):
    \begin{enumerate}
        \item The sequence $(a_n)_{n=m}^{\infty}$ converges to $L$.
        \item Every subsequence of $(a_n)_{n=m}^{\infty}$ converges to $L$.
    \end{enumerate}
    }
\end{proposition}

\begin{proposition}{A1.6.6.6}
    (Subsequences related to limit points) \emph{Let $(a_n)_{n=m}^{\infty}$ be a
    sequence of real numbers, and let $L$ be a real number. Then the following two
    statements are logically equivalent.
    \begin{enumerate}
        \item $L$ is a limit point of $(a_n)_{n=m}^{\infty}$.
        \item There exists a subsequence of $(a_n)_{n=m}^{\infty}$ which converges
        to $L$.
    \end{enumerate}
    }
\end{proposition}

\begin{theorem}{A1.6.6.8}
    (Bolzano-Weierstrass theorem) \emph{Let $(a_n)_{n=0}^{\infty}$ be a bounded
    sequence (i.e., there exists a real number $M > 0$ such that $|a_n| \leq M$
    for all $n \in \naturals$). Then there is at least one subsequence of
    $(a_n)_{n=0}^{\infty}$ which converges.}
\end{theorem}

\subsection{Real exponentiation, part II}
\begin{lemma}{A1.6.7.1}
    (Continuity of exponentiation) \emph{Let $x > 0$, and let $\alpha$ be a real
    number. Let $(q_n)_{n=0}^{\infty}$ be a sequence of rational numbers converging
    to $\alpha$. Then $(x^{q_n})_{n=0}^{\infty}$ is also a convergent sequence.
    Furthermore, if $(q'_n)_{n=0}^{\infty}$ is any other sequence of rational numbers
    converging to $\alpha$, then $(x^{q'_n})_{n=0}^{\infty}$ has the same limit
    as $(x^{q_n})_{n=0}^{\infty}$.}
\end{lemma}

\begin{definition}{A1.6.7.2}
    (Exponentiation to a real exponent) Let $x > 0$ be real, and let $\alpha$ be
    a real number. We define the quantity $x^\alpha$ by the formula $x^\alpha := \limit{x^{q_n}}$,
    where $(q_n)_{n=0}^{\infty}$ is any sequence of rational numbers converging to $\alpha$.
\end{definition}

\begin{proposition}{A1.6.7.3}
    \emph{All the results of Lemma 5.6.9, which held for rational numbers $q$ and
    $r$, continue to hold for real numbers $q$ and $r$.}
\end{proposition}
